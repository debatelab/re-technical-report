\documentclass{article}
\usepackage[x11names, rgb]{xcolor}
\usepackage[<<textencoding>>]{inputenc}
\usepackage{tikz}
\usetikzlibrary{arrows,shapes}
\usetikzlibrary{decorations.pathmorphing}
\usetikzlibrary{decorations.markings}

\usepackage{amsmath}
<<startpreprocsection>>%
\usepackage[active,auctex]{preview}
<<endpreprocsection>>%
<<gvcols>>%
<<cropcode>>%
<<docpreamble>>%

%\definecolor{attackColor}{RGB}{219,6,10}
%\definecolor{supportColor}{RGB}{50,144,24}

%\tikzstyle{argument}= [rectangle,rounded corners,draw=black, top color=white, bottom color=black!10, thick, text centered]
% for using linebreaks with `\\` we use `align = center` (see: https://tex.stackexchange.com/questions/123671/manual-automatic-line-breaks-and-text-alignment-in-tikz-nodes#124114
\tikzstyle{argument}= [rectangle,rounded corners,draw=black, top color=white, bottom color=black!10, thick, align=center]

%\tikzstyle{attack}= [>=latex',  shorten >=1pt, thick, auto, text=black, decorate, decoration = {snake, amplitude=1pt, segment length=4pt, pre length=2pt,post length=6pt}]
\tikzstyle{attack}= [>=latex',  shorten >=1pt, thick, auto, text=black, decorate, decoration = {snake, amplitude=0.5pt, segment length=4pt, pre length=2pt,post length=6pt}]
\tikzstyle{support}= [>=latex',  shorten >=1pt, thick, auto, text=black]

\begin{document}
\pagestyle{empty}
%
<<startpreprocsection>>%
<<preproccode>>
<<endpreprocsection>>%
%
<<startoutputsection>>
\enlargethispage{100cm}
% Start of code
% \begin{tikzpicture}[anchor=mid,>=latex',join=bevel,<<graphstyle>>]
\begin{tikzpicture}[>=latex',join=bevel,<<graphstyle>>]
\pgfsetlinewidth{1bp}
<<figpreamble>>%
<<drawcommands>>
<<figpostamble>>%
\end{tikzpicture}
% End of code
<<endoutputsection>>
%
\end{document}
%
<<startfigonlysection>>
\begin{tikzpicture}[>=latex,join=bevel,<<graphstyle>>]
\pgfsetlinewidth{1bp}
<<figpreamble>>%
<<drawcommands>>
<<figpostamble>>%
\end{tikzpicture}
<<endfigonlysection>>
